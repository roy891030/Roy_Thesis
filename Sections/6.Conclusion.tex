\chapter{結論}
\label{ch:conclusion}

\section{未來研究}
\label{sec:future_work}

本研究目前的主要後續工作是先修正程式流程中的關鍵問題,再重新評估模型效能,並與 0050 以一致口徑比較。重點如下。

\begin{enumerate}
  \item \textbf{DMFM 訓練訊號梯度中斷}\\
  問題:IC 與 factor return 相關項在計算時轉成常數,未真正參與反向傳播。\\
  修正:將損失改為全可微分實作,確保 supervised 訊號能有效更新模型參數。

  \item \textbf{資料切分造成測試洩漏}\\
  問題:目前用測試集做 early stopping,並同時回報最終成績。\\
  修正:改為 train/validation/test 三段切分,validation 僅用於調參與 early stopping,test 僅保留最終一次評估。

  \item \textbf{GAT 損失設計扭曲預測尺度}\\
  問題:現有損失含鼓勵預測方差擴大的項,容易導致輸出振幅不穩。\\
  修正:移除或弱化該項,重新平衡 correlation 與誤差項權重,提升訓練穩定性。

  \item \textbf{跨流程設定未一致貫穿}\\
  問題:如 \texttt{train\_ratio} 在訓練與評估流程未完全一致,影響跨模型公平比較。\\
  修正:統一參數在 artifact 建置、訓練、評估與回測全流程一致使用。

  \item \textbf{圖結構與評估框架仍可強化}\\
  問題:目前 GAT 實作未完整使用 universe graph,且指標解讀仍偏重 MSE。\\
  修正:導入完整市場關係圖,並以 IC/ICIR 與投組績效作為主要判讀依據。
\end{enumerate}

整體而言,後續研究將先完成上述程式修正,再以一致且可重現的評估流程重新驗證 DMFM 與 GAT 的實際增益。
