\chapter{緒論}
\label{ch:intro}

% 本章介紹研究背景、問題、方法與貢獻,以及論文架構

\section{研究背景}
\label{sec:background}

% 在此撰寫:
% - 股票市場預測的重要性
% - 傳統量化投資方法的演進
% - 深度學習在金融領域的應用趨勢
% - 圖神經網路的興起與優勢

(此處撰寫研究背景內容)

\section{研究問題}
\label{sec:problem}

% 在此撰寫:
% - 現有股票預測方法的限制
% - 市場結構與股票關聯性的重要性
% - 需要解決的核心問題

(此處撰寫研究問題內容)

\section{研究方法與貢獻}
\label{sec:contribution}

% 在此撰寫:
% - 本研究採用的方法(GAT、DMFM)
% - 主要創新點
% - 預期貢獻

(此處撰寫研究方法與貢獻內容)

\section{論文架構}
\label{sec:organization}

% 在此撰寫:
本論文共分為五章,各章內容如下:

第一章為緒論,說明研究背景、研究問題、研究方法與貢獻,以及論文架構。

第二章為文獻回顧,探討技術指標與特徵工程、LSTM/RNN 方法、圖神經網路基礎,以及 GAT 在股票預測的應用。

第三章為實驗設計,介紹資料來源與預處理、特徵工程與圖結構、模型設計(包含 LSTM、GAT、DMFM 模型)、損失函數,以及評估指標。

第四章為實驗結果,呈現模型對比、投組績效驗證,以及特徵重要性排名。

第五章為結論,總結主要發現、與既有研究的對比、模型限制,以及實踐應用與未來方向。
