\chapter{緒論}
\label{ch:intro}

\section{研究背景}
\label{sec:background}

股票市場的價格形成同時受到公司基本面、產業景氣、整體市場風險偏好與短期交易行為等因素影響,使得報酬序列呈現高雜訊、非線性與時變(time-varying)特性。在量化投資(quantitative investment)實務中,常以可計算的市場訊號作為決策依據,其中技術指標(Technical Indicators, TIs)透過歷史價格與成交量序列,將趨勢、動能與波動等資訊轉換為量化特徵,長期被廣泛用於特徵工程(feature engineering)與交易策略建構。

近年深度學習(deep learning)在金融預測任務上的應用逐漸普及,主要優勢在於能以資料驅動方式學習非線性映射,並在高維特徵下自動萃取有效表示。然而,當技術指標與衍生特徵數量擴張時,特徵集合往往伴隨高度相關、冗餘與尺度不一致,導致模型訓練成本提高、收斂不穩定,並可能削弱泛化能力。換言之,研究焦點逐步由「是否使用技術指標」轉向「如何將技術指標組織為深度模型可有效吸收的表示」。

另一方面,金融市場並非由彼此獨立的資產所構成。台灣股票之間普遍存在產業層級與市場層級的共同因子影響(common factors),使個股報酬在截面(cross-section)上呈現結構性相依。傳統時間序列模型(如 LSTM)多以單一資產的歷史序列為主要訊號來源,較難直接刻畫股票之間的關係拓撲;因此近年研究開始引入圖神經網路(Graph Neural Networks, GNNs),以「節點(股票)—邊(關係)」的方式建模產業或市場連結,並透過訊息傳遞(message passing)與注意力機制(attention mechanism)聚合鄰居資訊,以提升截面預測與排序任務的表現與可解釋性。

在上述脈絡下,結合多因子表示(multi-factor representation)與圖注意力(graph attention)的模型架構,特別是能分離產業與全市場共同影響的設計,提供了一條兼顧「特徵可學習性」與「關係結構建模」的路徑。本文後續將採用深度多因子模型結合圖注意力之框架,並以台灣日頻資料進行實證檢驗。

\section{研究問題}
\label{sec:problem}

綜合既有研究與實務需求,本文關注的核心問題可整理如下。

第一,技術指標所形成的高維特徵集合雖能涵蓋多面向市場訊號,但亦容易引入冗餘與尺度差異,使模型難以穩定學習;因此需要一套可重複、可檢核的特徵建置與整理流程,將價格與量能序列轉換為適合深度模型輸入的特徵表示,並降低不必要的噪音干擾。

第二,股票之間存在產業共通訊號與市場共同波動,僅依賴單一股票的時間序列特徵,可能無法充分利用截面關係資訊;因此需要能顯式建模股票關係的學習架構,使模型同時吸收「個股特徵」與「關係拓撲」兩類訊號。

第三,即便引入圖結構,模型表現仍高度仰賴圖的構建方式與關係品質;若未能有效分離產業與全市場層級的共同影響,模型可能將「產業押注」或「市場曝險」誤判為選股能力。因而本文特別關注:是否能透過分層中性化(hierarchical neutralization)的方式,將產業與市場影響自特徵表示中拆解,進而提升預測訊號的穩健性。

第四,在金融預測情境下,模型品質不宜僅以單一誤差指標衡量;更關鍵的是模型在截面排序上的有效性與穩定性。因此本文採用資訊係數(Information Coefficient, IC)與其穩健性指標 ICIR(Information Coefficient Information Ratio, ICIR)作為主要評估基礎,並輔以投資組合績效指標進行整體驗證。

\section{研究方法與貢獻}
\label{sec:contribution}

本文採用台灣股票日頻資料進行實證。資料主要來源為 TEJ 台灣經濟新報(Taiwan Economic Journal, TEJ),並匯出為 CSV 作為後續處理之輸入。以「交易日\(\times\)股票」為觀測單位,保留日期、股票代碼、OHLC、成交量/成交值、市值與產業分類等欄位;技術特徵則由價格與量能序列計算而得,其中 RSI、MACD、KD 等指標透過 \texttt{TA-Lib} 套件產生,最終形成固定的 54 維特徵集合供模型輸入(詳見第三章的變數表格與附錄清單)。

在關係建模方面,本文依第三章之設計建構兩類圖結構:(1)產業圖(Industry Graph),以同產業股票完全連結並加入自環;(2)全市場圖(Universe Graph),以完全圖表示整體市場共同因子影響。圖結構以 PyTorch Geometric(PyG)的 \texttt{edge\_index} 形式保存,並與特徵張量共同形成可重複使用的訓練產物(artifacts)。

模型設計以 Deep Multi-Factor Model(DMFM)為主體,透過「特徵編碼 \(\rightarrow\) 產業中性化 \(\rightarrow\) 全市場中性化 \(\rightarrow\) 階層式拼接 \(\rightarrow\) 深度因子學習 \(\rightarrow\) 因子注意力」的流程,在每個交易日的截面上輸出各股票之預測訊號;同時以較簡化的圖模型(GATRegressor)作為基準模型,進行可比性的對照實驗。評估上以 IC、Daily IC、ICIR 與產業中性 IC 作為核心預測指標,並以投資組合回測指標(年化報酬率、Sharpe ratio、勝率等)檢驗策略層級的有效性;資料切分採時間序列切分以避免資訊洩漏,並設計不同樣本長度(短期/中期/長期)檢驗穩健性。

基於上述方法設計,本文的主要貢獻歸納如下:
\begin{itemize}
    \item \textbf{資料與特徵流程的可重現性:}以 TEJ 日頻資料為基礎,建立從原始欄位到 54 維技術特徵(含 TA-Lib 指標)的完整特徵建置與張量化流程,並以 artifacts 形式保存,便於重複訓練與評估。
    \item \textbf{關係結構的分層建模:}同時建構產業圖與全市場圖,並在模型中引入分層中性化機制,以拆解產業與市場共同影響,提升截面預測訊號的穩健性與可解釋性。
    \item \textbf{以排序導向指標為核心的實證驗證:}以 IC/ICIR 與產業中性 IC 作為主要評估基礎,並搭配投資組合績效指標,提供模型在台灣市場情境下的多面向驗證結果。
\end{itemize}

\section{論文架構}
\label{sec:organization}

本文共分為五章,各章內容如下。

第一章為緒論,說明研究動機與背景、研究問題、研究方法與預期貢獻,並概述全文架構。

第二章為文獻回顧,依序整理技術指標在深度學習中的特徵表示、時間序列模型(如 LSTM)於金融預測的典型做法、圖神經網路的基本概念與代表性架構,以及圖注意力模型(GAT)在股票預測/排序任務上的應用與限制,作為後續方法設計之理論與實證依據。

第三章為實驗設計,說明資料來源與預處理流程、特徵工程與圖結構建構方式、模型架構(基準模型 GATRegressor 與 DMFM)、損失函數設定,以及評估指標與實驗切分規則。

第四章為實驗結果,呈現不同模型之預測表現比較、IC/ICIR 與產業中性 IC 之結果分析,並以投資組合績效驗證模型訊號在策略層面的有效性與穩健性(含不同樣本長度設定之比較)。

第五章為結論,總結本文之主要發現,討論方法限制與可能的改進方向,並提出後續在實務應用與研究延伸上的建議。
