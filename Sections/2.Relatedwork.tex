\chapter{文獻回顧}
\label{ch:relatedwork}

% 本章回顧技術指標、傳統機器學習、深度學習與圖神經網路在股票預測的相關研究

\section{技術指標與特徵工程}
\label{sec:technical_indicators}

技術指標在股票預測領域扮演重要角色,透過分析歷史價格與成交量資料來預測未來股價走勢。常見的技術指標包括移動平均線(Moving Averages, MA)、相對強弱指標(Relative Strength Index, RSI)、平滑異同移動平均線(Moving Average Convergence Divergence, MACD)以及能量潮指標(On-Balance Volume, OBV)等,這些指標分別用於捕捉市場趨勢、動能以及市場強度,對投資決策具有關鍵影響 \cite{delgadillo2024,ming2022}。

技術分析的核心在於整合多個指標以建立全面性的預測模型。實證研究顯示,結合多個技術指標的預測能力顯著優於單一指標 \cite{trancar2015,adeduro2023}。這種多指標整合的方法能夠從不同角度捕捉市場訊號,提升預測的穩健性與準確度。

近年來,深度學習技術的興起為技術指標的應用帶來新的突破。長短期記憶網路(Long Short-Term Memory, LSTM)等深度學習模型被廣泛應用於分析技術指標所產生的時間序列資料 \cite{konur2024,libastos2020}。LSTM 網路特別擅長捕捉資料中的長期依賴關係,使其適合處理股價的高度波動性。綜合分析指出,將深度學習技術與多個技術指標結合,可以顯著提升預測效能,其準確率能夠超越傳統基準 \cite{josevarshini2024,agrawal2022}。

此外,機器學習在技術指標參數優化方面也展現出顯著成效。Aguirre 等人證明,將遺傳演算法(Genetic Algorithms)應用於 MACD 指標的參數調整,能夠產生在各種市場條件下顯著優於傳統技術分析的投資策略 \cite{aguirre2020}。這表明機器學習不僅能改善交易策略的參數設定,還能增強其對市場動態變化的適應能力。

在投資組合管理方面,結合基本面分析與技術分析的混合模型已成為主流趨勢。Ouazzane 等人提出整合機器學習、財務比率與技術指標的混合模型,專門針對 Nasdaq 半導體市場進行投資組合優化,證明了機器學習在整合不同分析方法以建立穩健交易策略的有效性 \cite{ouazzane2025}。

情緒分析的整合也逐漸受到重視。將財經新聞的市場情緒納入預測模型,已被證明能夠提升股價預測的準確性,提供純粹技術指標可能忽略的額外市場脈絡 \cite{yan2023,guravkotrappa2020}。這顯示出混合模型的趨勢,即融合傳統技術分析、機器學習演算法與情緒分析,以建立更為穩健的預測框架。

綜上所述,將技術指標整合於先進的計算框架(如深度學習模型)中,為股價預測提供了有力的方法。多個互補指標的結合提升了預測準確性,而情緒分析的納入則引入了能夠影響市場行為與決策的重要維度。這種多面向的分析強調了對技術指標與新興機器學習方法的深入理解,對於在動態的股票市場預測領域中取得成功至關重要。

\section{LSTM/RNN 方法}
\label{sec:lstm_rnn}

長短期記憶網路(Long Short-Term Memory, LSTM)因其處理金融時間序列資料中複雜時間依賴關係的能力,在股票報酬率預測領域獲得廣泛關注。研究指出,LSTM 的內在架構能有效建模非線性金融資料,使其相較於傳統模型具有顯著優勢 \cite{ji2021}。LSTM 網路在處理股票市場指數的非線性特徵方面表現優異,其成功源於能夠從大量輸入維度中學習,而無需進行大量的特徵降維處理 \cite{joseph2023}。

近期研究強調整合額外預測因子(如市場情緒)的重要性。Wang 等人證明,將情緒分析整合至 LSTM 模型能顯著提升預測準確性,他們利用 Granger 因果檢定建立情緒指數與股票報酬率之間的關聯 \cite{li2024}。Yan 等人進一步證實,當 LSTM 模型結合多種資料來源(包括新聞情緒)時,預測效能表現優異 \cite{yan2023}。這表明豐富 LSTM 的輸入資料能夠改善預測效能,強調了在此類模型中建立全面特徵集的必要性 \cite{mahboob2023}。

對傳統 LSTM 架構的改良也成功提升了預測能力。Peng 與 Guo 在 LSTM 網路中引入結構變化,包括調整層數與神經元配置,這些改良提高了中國銀行股票的預測精度 \cite{pengguo2022}。同樣地,將 LSTM 與其他演算法結合的技術——如粒子群最佳化(Particle Swarm Optimization)以及 GARCH-LSTM 等混合模型——已被證明能有效處理資料波動並增強預測能力 \cite{sadon2024}。這些混合模型凸顯了 LSTM 作為推進股票預測方法論基礎要素的多樣性。

特徵工程技術如主成分分析(Principal Component Analysis, PCA)也被證明能提升 LSTM 的效能。PCA 有助於解決維度問題,同時保留資料的關鍵特徵 \cite{mi2023}。將 PCA 與 LSTM 結合,研究者在多個金融市場的預測準確度上取得了顯著進展 \cite{wen2020}。這種方法不僅降低了計算複雜度,還能保持模型對重要市場特徵的敏感性。

此外,LSTM 模型的架構設計也持續演進。研究顯示,透過調整 LSTM 的層數、隱藏單元數量以及dropout 機制,能夠顯著影響模型的預測表現 \cite{pengguo2022}。這些架構上的微調使得 LSTM 能夠更好地適應不同市場特性與資料特徵,展現出其作為深度學習模型的靈活性。

綜上所述,LSTM 在股票報酬率預測的應用研究揭示了一個能夠捕捉金融資料中複雜模式與關係的穩健框架。持續的建模技術改良——包括情緒整合、混合架構以及特徵優化——進一步鞏固了 LSTM 作為金融預測關鍵工具的地位。這些進展不僅提升了預測準確性,也為理解金融市場的動態行為提供了更深入的見解。

\section{圖神經網路基礎}
\label{sec:gnn_basics}

% 在此撰寫:
% - 圖神經網路的基本概念
% - GCN (Graph Convolutional Network)
% - GAT (Graph Attention Network)
% - 圖結構在金融領域的應用

(此處撰寫圖神經網路基礎內容)

\section{GAT 在股票預測的應用}
\label{sec:gat_stock}

% 在此撰寫:
% - GAT 在股票預測的相關研究
% - 產業圖與市場圖的建構方法
% - 現有方法的限制與改進空間
% - 與本研究的關聯

(此處撰寫 GAT 在股票預測的應用內容)
