\chapter{文獻回顧}
\label{ch:relatedwork}

本章依序回顧:(1)技術指標如何轉化為深度學習可用之特徵表示,(2)LSTM 等時間序列模型在金融預測中的典型做法與延伸,(3)圖神經網路的基本概念與代表性架構,(4)GAT 及其變形在股票預測/排序任務上的應用與限制,並據此銜接本研究後續以多因子與圖注意力為核心之模型設計與實驗流程。

\section{技術指標如何轉化為深度學習可用之特徵表示}
\label{sec:technical_representation}

技術指標(Technical Indicators, TIs)以歷史價格與成交量為基礎,將市場的趨勢、動能與波動等訊號轉換為可計算的量化特徵,長期被用於股票預測任務的特徵工程。當指標種類擴張時,特徵集合往往同時伴隨高度相關、冗餘與尺度不一致,使模型訓練成本上升,並可能影響泛化能力與穩健性。

為回應上述挑戰,研究焦點逐步由「是否使用技術指標」轉向「如何將技術指標組織為深度學習可有效吸收的特徵表示」。Agrawal 等人以技術指標作為深度學習模型的輸入,並在股價預測任務中強調多指標組合能提供更完整的市場訊號。\cite{agrawal2022stock}
Agrawal 等人亦提出以最佳化的深度學習架構搭配技術指標的預測流程,指出在輸入維度增加時,指標集合的挑選與組織方式將直接影響模型是否能有效吸收訊號並降低冗餘干擾。\cite{agrawal2019stock}
此外,Li 與 Bastos 以系統性綜述方式整理「技術分析+深度學習」在股市預測的主要設計與趨勢,並指出「如何識別最優指標集」仍是領域內的重要議題。\cite{li2020stock}

總結而言,技術指標在深度學習框架中的角色可被視為「可學習特徵空間的原料」:研究重點不僅在於指標本身,而在於如何以更合適的表示方式降低冗餘並保留有效訊號,進而提升後續預測任務的可學習性與穩定性。


% 若您的主檔(main.tex)尚未載入,請在 preamble 加上:
% \usepackage{graphicx}

\section{LSTM 等時間序列模型在金融預測中的典型做法與延伸}

時間序列模型在金融預測中主要利用價格與特徵序列的時間依賴性,學習由過去資訊推估未來表現。長短期記憶網路(Long Short-Term Memory, LSTM)透過閘門(gates)機制控制資訊流動,能緩解長序列學習的梯度消失(vanishing gradients)問題,因此成為金融時間序列建模的代表性方法之一。

\begin{figure}[htbp]
  \centering
  \includegraphics[width=0.92\textwidth]{img/LSTM_graph.jpg}
  \caption{LSTM 單元的資訊流(Forget / Update / Output)。其中 $x_t$ 為當期輸入,$h_{t-1}$ 為前一期隱狀態(hidden state),$c_{t-1}$ 為前一期記憶狀態(cell state)。}
  \label{fig:lstm_cell_flow}
\end{figure}

如圖 \ref{fig:lstm_cell_flow} 所示,LSTM 會沿著記憶狀態(cell state)主幹 $c_{t-1}\rightarrow c_t$ 保留長期資訊,並透過三個閘門控制「遺忘(Forget)」、「更新(Update)」與「輸出(Output)」:
(1) 遺忘閘門 $f_t$ 決定保留多少過去記憶 $c_{t-1}$;
(2) 更新步驟由輸入閘門 $i_t$ 與候選記憶 $g_t$(圖中 memory cell)共同決定要寫入多少新資訊;
(3) 輸出閘門 $o_t$ 決定由當期記憶 $c_t$ 產生多少輸出到隱狀態 $h_t$。
對應的計算可寫為:
\begin{equation}
\begin{aligned}
f_t &= \sigma\!\left(W_f [h_{t-1}, x_t] + b_f\right), \\
i_t &= \sigma\!\left(W_i [h_{t-1}, x_t] + b_i\right), \\
g_t &= \tanh\!\left(W_g [h_{t-1}, x_t] + b_g\right), \\
c_t &= f_t \odot c_{t-1} + i_t \odot g_t, \\
o_t &= \sigma\!\left(W_o [h_{t-1}, x_t] + b_o\right), \\
h_t &= o_t \odot \tanh(c_t),
\end{aligned}
\label{eq:lstm_gates_aligned}
\end{equation}
其中 $\sigma(\cdot)$ 為 sigmoid 函數、$\tanh(\cdot)$ 為雙曲正切函數,$\odot$ 表示逐元素相乘(element-wise multiplication)。圖中的方形節點可視為逐元素乘法與加總的運算節點,對應到 $f_t \odot c_{t-1}$ 與 $i_t \odot g_t$ 兩條路徑匯入 $c_t$ 的更新。

在實務與研究中,LSTM 的延伸方向常包含多目標預測、結合注意力機制(attention mechanism)與去噪/重加權策略,以及高頻資料下的深層序列建模。Zaheer 等人設計多參數預測架構以同時預測不同價格維度,並比較不同深度模型配置在特定資料規模下的表現差異。\cite{zaheer2023multi}
Qiu 等人將注意力機制引入 LSTM 架構,以在時間維度上動態調整特徵權重,強化模型在金融資料高雜訊情境下的預測能力。\cite{qiu2020forecasting}
在高頻(例如 5 分 K)資料情境下,鄭邦廷以疊層式 LSTM(Stacked LSTM)捕捉更複雜的非線性時間依賴,用於買賣點預測。\cite{cheng2023stock}
此外,廖俊翔透過自相關分析與特徵篩選納入跨市場外部特徵,並展示外部訊號在一定程度上能降低預測誤差。\cite{liao2022applying}

整體而言,時間序列方法能有效吸收單一資產(或單一特徵集合)的歷史資訊,但其多數設計仍以「序列本身」為主要訊號來源,較難直接刻畫股票之間的結構性關係與共通影響;因此後續研究開始引入圖結構以建模市場關係。

\section{圖神經網路的基本概念與代表性架構}
\label{sec:gnn_basics}

圖神經網路(Graph Neural Networks, GNNs)以圖作為基本資料結構,透過節點(node)與邊(edge)描述實體及其關係,並以訊息傳遞(message passing)機制聚合鄰居資訊,學得具備拓撲語意的節點表示。Wu 等人對 GNN 架構進行系統性整理,並區分頻域(spectral-based)與空域(spatial-based)圖卷積之主要差異,同時歸納多類代表性 GNN 架構族群。\cite{wu2020comprehensive}

在金融市場中,股票之間存在產業層級與市場層級的共同因子影響,亦可能呈現共動性、傳染效應與結構性相依關係。以圖結構建模股票關係,可使模型同時利用「個股特徵」與「關係拓撲」來提升截面預測或排序任務的表現,亦為後續在模型中引入中性化或分離共通影響提供方法基礎。

\section{GAT 及其變形在股票預測/排序任務上的應用與限制}
\label{sec:gat_stock}

圖注意力網路(Graph Attention Network, GAT)在鄰居聚合框架中引入注意力機制,使模型可對不同鄰居節點賦予不同權重,提升在異質或噪音關係下的表徵能力與可解釋性。以注意力係數為例,其形式可寫為:
\begin{equation}
\alpha_{ij} = \mathrm{softmax}_j\Big(\mathrm{LeakyReLU}\big(a^\top [W h_i \, \Vert \, W h_j]\big)\Big),
\label{eq:gat_attention}
\end{equation}
其中 $h_i$ 為節點表示,$\alpha_{ij}$ 代表節點 $i$ 對鄰居 $j$ 的注意力權重。

在股票預測任務中,GAT 類方法的效能高度依賴圖結構建構方式與關係訊號品質。Huang 等人提出多層級圖注意力模型(ML-GAT),以分層注意力機制分別處理節點狀態與關係類型,使模型能在多來源關係下更細緻地擷取有效訊號。\cite{huang2022mlgat}
Song 等人以股票價格與股票關係資訊進行圖聚合式排序預測,並討論關係資訊設計對截面排序表現的影響。\cite{song2023stock}
當市場關係稀疏或不完整時,Cheng 等人提出多特徵圖注意力網路,透過整合多面向特徵訊號來提升股票預測能力。\cite{cheng2024stock}

然而,GAT 類方法仍可能面臨下列限制:其一,注意力權重在噪音邊存在時可能分散,增加過擬合風險;其二,圖建構規則、關係來源與連邊密度高度敏感,使模型可轉移性與穩健性需要透過更嚴謹的實證評估驗證;其三,靜態圖難以全面反映關係隨時間變動的特性,可能降低模型對市場結構變化的敏感度。

基於上述研究脈絡,Wei 等人提出以深度多因子模型結合圖注意力的框架,並透過分離產業與全市場層級影響來提升因子信號的可用性與穩健性。\cite{wei2022factor}
本研究將沿用此一方向,在後續方法設計中以多層級關係建模與中性化概念為核心,並以 IC/ICIR 等指標作為實證評估依據,以對齊第三章所述之實驗設計與評估流程。
