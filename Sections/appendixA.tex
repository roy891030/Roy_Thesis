\chapter{詳細表格與補充圖表}
\label{appendix:tables_figures}

% 附錄 A:放置論文中引用但篇幅較大的表格與圖表

\section{完整實驗數據表}
\label{appendix:full_results}

% 在此放置完整的實驗數據表格
% 例如:所有股票的預測結果、每日回測數據等

(此處放置詳細表格內容)

% 範例表格:
% \begin{table}[h]
% \centering
% \caption{完整模型預測結果(所有股票)}
% \begin{tabular}{c|c|c|c|c}
% \toprule[1.25pt]
% 股票代碼 & 實際報酬率 & LSTM預測 & GAT預測 & DMFM預測 \\
% \midrule
% 2330 & 2.5\% & 2.1\% & 2.3\% & 2.4\% \\
% 2317 & 1.8\% & 1.5\% & 1.7\% & 1.8\% \\
% ... & ... & ... & ... & ... \\
% \bottomrule[1.25pt]
% \end{tabular}
% \label{table:full_predictions}
% \end{table}

\section{補充圖表}
\label{appendix:additional_figures}

% 在此放置補充的圖表
% 例如:不同時期的績效曲線、特徵分布圖等

(此處放置補充圖表內容)

% 範例圖表:
% \begin{figure}[htb]
%   \centering
%   \includegraphics[width=0.9\textwidth]{img/feature_distribution.png}
%   \caption{技術指標特徵分布圖}
%   \label{fig:feature_distribution}
% \end{figure}

\section{不同參數設定的實驗結果}
\label{appendix:hyperparameter_tuning}

% 在此放置超參數調整的詳細結果

(此處放置超參數實驗結果)
