\chapter{實驗設計}
\label{ch:experiment_design}

% 本章介紹實驗的完整設計,包含資料、特徵、模型、損失函數與評估指標

\section{資料來源與預處理}
\label{sec:data}

% 在此撰寫:
% - 資料來源(台股/美股、時間範圍)
% - 資料欄位(OHLCV、技術指標等)
% - 預處理步驟(缺失值處理、標準化等)
% - 訓練/驗證/測試集劃分

(此處撰寫資料來源與預處理內容)

\section{特徵工程與圖結構}
\label{sec:features}

% 在此撰寫:
% - 技術指標特徵
% - 產業圖的建構方法
% - 市場圖的建構方法
% - 圖的鄰接矩陣設計

(此處撰寫特徵工程與圖結構內容)

\section{模型設計}
\label{sec:models}

本研究設計三種模型進行對比實驗:LSTM 模型、GAT 模型與 DMFM 提案模型。

\subsection{LSTM 模型}
\label{subsec:lstm_model}

% 在此撰寫:
% - LSTM 架構設計
% - 輸入輸出格式
% - 超參數設定

(此處撰寫 LSTM 模型內容)

\subsection{GAT 模型}
\label{subsec:gat_model}

% 在此撰寫:
% - GAT 架構設計
% - 注意力機制
% - 多頭注意力設定
% - 超參數設定

(此處撰寫 GAT 模型內容)

\subsection{DMFM 提案模型}
\label{subsec:dmfm_model}

% 在此撰寫:
% - DMFM (Dynamic Multi-Factor Model) 架構
% - 與 GAT 的差異
% - 動態因子選擇機制
% - 超參數設定

(此處撰寫 DMFM 提案模型內容)

\section{損失函數}
\label{sec:loss}

% 在此撰寫:
% - 使用的損失函數(MSE, MAE 等)
% - 損失函數的選擇理由
% - 優化器與學習率設定

(此處撰寫損失函數內容)

\section{評估指標}
\label{sec:metrics}

% 在此撰寫:
% - 預測準確度指標(RMSE, MAE, R² 等)
% - 投資組合績效指標(Sharpe Ratio, 年化報酬率、最大回撤等)
% - 評估方法說明

(此處撰寫評估指標內容)
