% 請確保在 main.tex 或 preamble 中已載入以下套件:
% \usepackage{algorithm}
% \usepackage{algpseudocode}
% \usepackage{amsmath}
% \usepackage{booktabs}

\chapter{實驗設計}
\label{ch:experiment_design}

本章詳細介紹實驗的完整設計,包含資料來源與預處理、特徵工程與圖結構建構、模型架構設計等。實驗旨在驗證 Deep Multi-Factor Model (DMFM) 在台灣股票市場預測的有效性,並與時間序列基準模型進行對比分析。

\section{資料來源與預處理}
\label{sec:data}

\subsection{資料來源}
本研究使用台灣股票市場資料,時間範圍涵蓋 2019 年第三季至 2025 年第三季,共計約 6 年的交易資料,涵蓋約 771 檔股票。資料集包含以下主要欄位:

\begin{itemize}
    \item \textbf{價格資料}: 開盤價 (Open)、最高價 (High)、最低價 (Low)、收盤價 (Close)、成交量 (Volume)
    \item \textbf{估值指標}: 本益比 (P/E Ratio)、股價淨值比 (P/B Ratio)、股價營收比 (P/S Ratio)
    \item \textbf{技術指標}: 移動平均線 (MA)、相對強弱指標 (RSI)、布林通道 (Bollinger Bands)、MACD 等
    \item \textbf{產業分類}: TEJ 產業代碼與名稱,用於建構產業關聯圖
\end{itemize}

資料來源包括台灣經濟新報 (TEJ) 資料庫與公開市場資訊,確保資料的準確性與完整性。

\subsection{資料預處理}

資料預處理是確保模型訓練品質的關鍵步驟,本研究採用的預處理流程如演算法 \ref{alg:preprocessing} 所示。

\begin{algorithm}[htbp]
\caption{資料預處理流程}
\label{alg:preprocessing}
\begin{algorithmic}[1]
\State 讀取原始股票資料 (OHLCV, 技術指標, 估值指標)
\State 移除缺失值以 0 填補 \Comment{避免資料洩漏}
\State 計算標籤: $y_t = \frac{P_{t+k} - P_t}{P_t}$ \Comment{$k$ 日未來報酬率}
\State 建構產業圖鄰接矩陣 $E_{ind}$
\State 建構全市場圖鄰接矩陣 $E_{uni}$
\State 儲存為 PyTorch tensor 格式
\State {特徵標準化由模型內部的 BatchNorm 完成}
\end{algorithmic}
\end{algorithm}

\subsubsection{缺失值處理}

為確保模型輸入的穩定性,本研究採用簡潔的缺失值處理策略:對所有缺失值以 0 填補。此方法在金融時間序列預測中廣泛使用,既能保留資料結構,又能避免引入額外假設。相較於複雜的插補方法,零填補不會造成資料洩漏 (data leakage),符合實際交易情境。

\subsubsection{特徵標準化}

不同於傳統做法在資料預處理階段進行 Z-score 標準化,本研究遵循 Wei et al. (2022) 的設計,在模型內部使用 \textbf{Batch Normalization} 進行特徵標準化:

\begin{equation}
x_{normalized} = \frac{x - \mu_{batch}}{\sqrt{\sigma^2_{batch} + \epsilon}}
\label{eq:batchnorm}
\end{equation}

其中 $\mu_{batch}$ 與 $\sigma^2_{batch}$ 為當前 mini-batch 的均值與變異數,$\epsilon$ 為數值穩定項 (通常設為 $10^{-5}$)。

此設計具有以下優勢:

\begin{itemize}
    \item \textbf{動態標準化}: 每個 batch 獨立計算統計量,適應市場環境變化
    \item \textbf{訓練穩定性}: BatchNorm 有助於梯度流動,加速收斂
    \item \textbf{正則化效果}: 訓練時的 mini-batch 隨機性提供額外正則化
    \item \textbf{等價性}: Wei et al. (2022) 指出,BatchNorm 在金融時間序列建模中等價於截面標準化 (cross-sectional normalization)
\end{itemize}

\subsubsection{標籤定義}

本研究預測未來 $k$ 日的報酬率,標籤定義為:

\begin{equation}
y_t = \frac{P_{t+k} - P_t}{P_t}
\label{eq:return}
\end{equation}

其中 $P_t$ 為時間 $t$ 的收盤價,$k$ 為預測視窗 (本研究設定為 5,即預測未來 5 日報酬率)。此定義方式符合金融實務,直接對應投資報酬率。

\subsection{資料集劃分}

考慮金融時間序列的時序特性,本研究採用時間序列劃分方式,將資料依 80:20 比例切分為訓練集與測試集:

\begin{itemize}
    \item \textbf{訓練集}: 前 80\% 時間點的資料 (約 1,168 個交易日),用於模型參數學習
    \item \textbf{測試集}: 後 20\% 時間點的資料 (約 292 個交易日),用於評估模型泛化能力
\end{itemize}

此劃分方式嚴格遵守時間順序,確保模型訓練時不會使用未來資訊 (look-ahead bias),符合實際交易情境。

\section{特徵工程與圖結構}
\label{sec:features}

\subsection{特徵選取與分類}

本研究選用 56 個技術指標與估值特徵,涵蓋動量、波動率、成交量、技術指標、統計特性等面向,詳細分類如表 \ref{tab:features} 所示。

\begin{table}[htbp]
\centering
\caption{特徵分類與說明}
\label{tab:features}
\begin{tabular}{llp{6cm}}
\toprule
\textbf{類別} & \textbf{代表特徵} & \textbf{說明} \\
\midrule
動量類 & ret\_1, ret\_5, ret\_10, ret\_20, mom\_diff\_10/20 & 不同時間窗口的價格報酬率與動量差 \\
移動平均 & px\_over\_sma\_5/10/20/60 & 價格相對於移動平均的偏離度 \\
波動率 & std\_ret\_5/10/20/60, atr\_14 & 報酬率標準差與平均真實波幅 \\
成交量 & vol\_over\_ma\_5/10/20/60 & 成交量相對於均值的偏離度 \\
技術指標 & rsi\_14, macd, stoch\_k/d & 常用技術分析指標 \\
統計特性 & skew\_20, kurt\_20, zscore\_close & 報酬率分布的偏度、峰度與 Z-score \\
反轉 & rev\_1/5/10 & 短期價格反轉指標 \\
極值 & roll\_max/min\_5/10/20/60 & 滾動窗口的最高/最低價 \\
流動性 & amihud\_5/20 & Amihud 非流動性指標 \\
估值 & pb, ps & 股價淨值比、股價營收比 \\
\bottomrule
\end{tabular}
\end{table}


\subsection{圖結構設計}

\subsubsection{產業圖 (Industry Graph)}

\subsubsection{全市場圖 (Universe Graph)}

\subsubsection{鄰接矩陣表示}

\subsection{圖結構的建構流程}


\section{模型設計}
\label{sec:models}

\subsection{LSTM 基準模型}
\label{subsec:lstm_baseline}

\subsubsection{模型架構}

\subsubsection{超參數配置}

\subsubsection{與 DMFM 的差異}

\subsection{DMFM 模型架構(Wei et al. 2022)}
\label{subsec:dmfm_model}

\subsubsection{整體架構概覽}

\subsubsection{步驟 1: 特徵編碼器}

\subsubsection{步驟 2: 產業中性化 (Industry Neutralization)}

\subsubsection{步驟 3: 全市場中性化 (Universe Neutralization)}

\subsubsection{步驟 4: 階層式特徵拼接}

\subsubsection{步驟 5: 深度因子學習}

\subsubsection{步驟 6: 因子注意力模組(解釋性)}

\subsubsection{超參數配置}

\section{損失函數}
\label{sec:loss}

\subsection{總損失函數}

\subsection{Information Coefficient (IC)}

\subsection{優化器設定}


\section{評估指標}
\label{sec:metrics}

\subsection{預測準確度指標}

\subsubsection{Information Coefficient (IC)}

\subsubsection{IC Information Ratio (ICIR)}

\subsubsection{產業中性 IC (Industry-Neutral IC)}

\subsubsection{方向準確率 (Directional Accuracy)}

\subsubsection{均方誤差指標 (MSE, RMSE, MAE)}

\subsection{投資組合績效指標}

\subsubsection{年化報酬率 (Annualized Return)}

\subsubsection{Sharpe Ratio}

\subsubsection{最大回撤 (Maximum Drawdown)}

\subsubsection{勝率 (Win Rate)}

\subsection{基準比較}

\subsubsection{LSTM 基準}

\subsubsection{台灣 50 ETF 基準}

\subsection{評估流程}