% 請確保在 main.tex 或 preamble 中已載入以下套件:
% \usepackage{algorithm}
% \usepackage{algpseudocode}
% \usepackage{amsmath}
% \usepackage{booktabs}

\chapter{實驗設計}
\label{ch:experiment_design}

本章詳細介紹實驗的完整設計,包含資料來源與預處理、特徵工程與圖結構建構、模型架構設計等。實驗旨在驗證 Deep Multi-Factor Model (DMFM) 在台灣股票市場預測的有效性,並與時間序列基準模型進行對比分析。

\section{資料來源與預處理}
\label{sec:data}

本研究採用單一日頻股票資料 CSV 檔作為輸入資料,資料包含每日價格與成交資訊,並同時內含產業分類欄位以利建構產業圖。程式支援中英文欄位對照,最少需具備日期、股票代碼與收盤價欄位,其餘欄位(開高低價、成交量/成交值、市值、流通股數、PB/PS 等)作為因子特徵使用。資料期間由參數 start\_date 與 end\_date 指定,並以交易日序列為主軸建立時間索引。若資料來源為 TEJ/TSE 等第三方資料庫,請在此節補充實際來源、授權或取得方式。【待補:資料來源、時間區間與股池說明】

\subsection{資料來源}
本研究使用【待補:資料來源名稱】之台灣股票市場日頻資料。資料表至少包含以下欄位:日期、股票代碼、收盤價;其餘可選欄位包含開高低價、成交量、成交值、市值、流通股數、PB/PS 與產業分類欄位(如 TEJ 產業名稱/代碼)。為確保可重現性,程式內已定義中英文欄位映射,當 CSV 欄位名稱不同時仍可正確辨識。

\subsection{資料預處理}

資料預處理是確保模型訓練品質的關鍵步驟,本研究採用的預處理流程如演算法 \ref{alg:preprocessing} 所示。

\begin{algorithm}[htbp]
\caption{資料預處理流程}
\label{alg:preprocessing}
\begin{algorithmic}[1]
\State 讀取原始股票資料 (OHLCV,成交量/成交值,估值指標,產業分類)
\State 依股票代碼與日期排序,數值欄位轉為數值型態
\State 計算標籤: $y_t = \frac{P_{t+k} - P_t}{P_t}$ \Comment{$k$ 日未來報酬率}
\State 建構產業圖鄰接矩陣 $E_{ind}$
\State 建構全市場圖鄰接矩陣 $E_{uni}$
\State 建立特徵張量 $F^t \in \mathbb{R}^{T \times N \times F}$ 與標籤張量 $y^t \in \mathbb{R}^{T \times N}$
\State 以 0 取代特徵中的 NaN/Inf,標籤 NaN 保留並以 mask 排除
\State {特徵標準化由模型內部的 BatchNorm 完成}
\end{algorithmic}
\end{algorithm}

\textbf{缺失值處理:}本研究不對缺失值進行插補。特徵層面,所有 NaN/Inf 以 0 取代(以保持張量完整性);標籤層面,NaN 標籤在訓練與評估時以有效樣本 mask 排除,使模型僅學習可觀測資料。此作法可避免以非實際市場資訊填補所造成的估計偏誤。

\textbf{特徵標準化:}資料前處理階段不進行截面標準化,特徵保留原始尺度。模型端在輸入層使用 \textbf{Batch Normalization} 進行正規化:

\begin{equation}
x_{normalized} = \frac{x - \mu_{batch}}{\sqrt{\sigma^2_{batch} + \epsilon}}
\label{eq:batchnorm}
\end{equation}

其中 $\mu_{batch}$ 與 $\sigma^2_{batch}$ 為當前 mini-batch 的均值與變異數,$\epsilon$ 為數值穩定項 (通常設為 $10^{-5}$)。

此設計具有以下優勢:

\begin{itemize}
    \item \textbf{動態標準化}: 每個 batch 獨立計算統計量,適應市場環境變化
    \item \textbf{訓練穩定性}: BatchNorm 有助於梯度流動,加速收斂
    \item \textbf{正則化效果}: 訓練時的 mini-batch 隨機性提供額外正則化
    \item \textbf{等價性}: BatchNorm 在金融時間序列建模中可視為截面標準化的近似
\end{itemize}

\textbf{標籤定義:}標籤定義為未來 $k$ 日報酬率,公式如下:

\begin{equation}
y_t = \frac{P_{t+k} - P_t}{P_t}
\label{eq:return}
\end{equation}

其中 $P_t$ 為時間 $t$ 的收盤價,$k$ 為預測視窗 (預設 $k=5$)。此定義屬於 forward return,避免使用當期或過去資料洩漏未來資訊。

\subsection{資料集劃分}

本研究採用時間序列切分,避免未來資訊洩漏。資料以時間順序切分為訓練集與測試集,預設比例為 80\%:20\%(前 80\% 為訓練、後 20\% 為測試)。訓練過程採用 early stopping,並以測試期 ICIR 作為模型保存依據。若需更嚴謹實驗,可進一步拆出 validation set 以獨立調參。【可選補充】

\section{特徵工程與圖結構}
\label{sec:features}

\subsection{特徵選取與分類}

本研究特徵涵蓋多種技術面與統計面訊號,主要可分為:
\begin{itemize}
    \item \textbf{報酬與動能特徵}: 1/3/5/10/20 日報酬、短長動能差 (ret10$-$ret1、ret20$-$ret1)。
    \item \textbf{趨勢特徵}: 價格相對移動平均 (5/10/20/60 日)、價格相對近 20 日高低點位置。
    \item \textbf{波動與風險特徵}: 5/10/20/60 日報酬標準差、ATR、最大回撤 (20 日)、beta 與特質波動。
    \item \textbf{量能特徵}: 成交量相對均值、上/下漲日平均量。
    \item \textbf{技術指標}: RSI、KD、MACD。
    \item \textbf{統計特徵}: 偏度、峰度、價格 z-score。
    \item \textbf{流動性與估值}: Amihud illiquidity、PB、PS。
\end{itemize}
上述特徵由單一 CSV 自動衍生,不需額外外部因子資料。


\subsection{圖結構設計與建構流程}
\textbf{產業圖 (Industry Graph):}產業圖以「同產業股票完全連結」為設計原則,將屬於同一產業的股票視為相互連結,並為每檔股票加入自環。此設計能捕捉同產業內的共通訊號與產業影響。

\textbf{全市場圖 (Universe Graph):}全市場圖採完全圖設計,所有股票相互連結並包含自環,用以表徵跨產業的市場共同因子影響。此設計對應 DMFM 論文中的 Universe-level 關係,能強化市場共同變動的抽取。

\textbf{鄰接矩陣表示:}圖結構以 PyG 的 edge\_index 表示,即 [2, E] 形式之邊索引。實作上先以鄰接矩陣建立連結,再轉換為邊索引格式,以利 GAT 層使用。

\textbf{建構流程:}由輸入 CSV 讀取產業欄位(如 TEJ 產業名稱/代碼),依股票代碼分組建立產業圖;若缺少產業欄,則預設所有股票屬於同一產業。全市場圖則依股票清單建立完全圖。圖結構與特徵張量一併存入 artifacts,供模型訓練與評估重複使用。


\section{模型設計}
\label{sec:models}

\subsection{LSTM 基準模型}
\label{subsec:lstm_baseline}

\textbf{模型架構:}目前程式碼未包含 LSTM 實作。若需保留此節,建議改為「GATRegressor 基準模型」,或自行補上 LSTM baseline。以下提供兩種處理方式:
\begin{itemize}
    \item 改寫為 GATRegressor 基準:兩層 GAT + 線性輸出,僅使用產業圖。
    \item 新增 LSTM baseline:以時間序列特徵輸入 LSTM,再用 MLP 輸出預測。
\end{itemize}
【待選:採用 GATRegressor 或補上 LSTM baseline】

\textbf{超參數配置:}若採 GATRegressor 基準,建議使用與 DMFM 相同的 hidden\_dim、heads 與 dropout,以確保可比性;若採 LSTM baseline,則需補充 LSTM 的 hidden size、層數、dropout 與輸出頭設定。【待補:基準模型超參數】

\textbf{與 DMFM 的差異:}LSTM 基準僅利用時間序列特徵,不顯式建模股票間關係;DMFM 則透過產業圖與全市場圖進行中性化,並以階層式拼接整合多層語境,因此能同時捕捉個股特徵與結構性訊號。【待補:是否保留 LSTM baseline】

\subsection{DMFM 模型架構(Wei et al. 2022)}
\label{subsec:dmfm_model}

DMFM 以「特徵編碼 $\rightarrow$ 產業中性化 $\rightarrow$ 全市場中性化 $\rightarrow$ 階層式拼接 $\rightarrow$ 深度因子學習 $\rightarrow$ 因子注意力」為主要流程。模型在每個交易日 $t$ 對股票截面進行運算,輸出每檔股票的預測因子值。

\textbf{步驟 1:特徵編碼器:}原始特徵 $F^t$ 經過 BatchNorm 進行截面標準化,再以 MLP 編碼為 $C^t$(hidden\_dim 維度),作為原始股票語境表示。

\textbf{步驟 2:產業中性化:}以產業圖為圖結構進行 GAT 運算,得到產業影響 $H_I^t$。產業中性特徵定義為:$C_I^t = C^t - H_I^t$,藉此移除產業共通影響。

\textbf{步驟 3:全市場中性化:}在產業中性特徵 $C_I^t$ 上,再以全市場圖進行 GAT 運算得到 $H_U^t$。全市場中性特徵定義為:$C_U^t = C_I^t - H_U^t$,用以移除市場共同影響。

\textbf{步驟 4:階層式特徵拼接:}將三層特徵拼接形成階層表示:$H^t = [C^t \, || \, C_I^t \, || \, C_U^t]$。此設計保留原始語境與兩層中性化資訊,提升模型表達力。

\textbf{步驟 5:深度因子學習:}拼接後的階層特徵輸入 MLP decoder,輸出深度因子 $f^t$(每檔股票 1 個數值),作為預測信號。

\textbf{步驟 6:因子注意力模組:}模型同時對原始特徵學習注意力權重,計算 $\hat{f}^t = F^t \odot a^t$,用以估計「哪些特徵驅動因子輸出」,並作為解釋與監督項。

\textbf{超參數配置:}預設配置如下:hidden\_dim=64、heads=2、dropout=0.1、epochs=200、learning rate=1e$-$4、weight decay=0.01、lambda\_attn=0.1、lambda\_ic=1.0、patience=30。

\section{損失函數}
\label{sec:loss}

\subsection{總損失函數}
本研究使用 DMFM 對應之損失:
\begin{equation}
L = \lambda_{attn} \cdot \lVert f - \hat{f} \rVert + \lambda_{IC} \cdot (1 - IC) - \lambda_b \cdot b
\label{eq:loss}
\end{equation}
其中 $\lVert f - \hat{f} \rVert$ 代表注意力估計誤差;$IC$ 為資訊係數;$b$ 為截面回歸得到的因子收益。程式中 $\lambda_b$ 固定為 0.01,以降低因子收益項之不穩定性。

\textbf{Information Coefficient (IC):}IC 以 Pearson correlation 計算,基於每個交易日的截面預測與真實報酬,衡量預測排序的有效性。

\textbf{優化器設定:}使用 AdamW 優化器,並加入 weight decay 以抑制過度擬合。


\section{評估指標}
\label{sec:metrics}

\subsection{預測準確度指標}
\textbf{IC / Daily IC / ICIR:}IC 為整體樣本的 Pearson correlation;Daily IC 為每交易日 IC 的平均值;ICIR 定義為 IC 平均值除以 IC 標準差,用以衡量資訊係數的穩定性。

\textbf{產業中性 IC:}【待補:若有計算產業中性 IC,請補充定義與計算方式;若未使用,可刪此項。】

\textbf{方向準確率:}方向準確率定義為預測方向與真實方向一致的比例,用以衡量模型對漲跌方向的掌握程度。

\textbf{誤差指標:}以全部測試樣本誤差衡量預測偏差,包含 MSE、RMSE 與 MAE。

\subsection{投資組合績效指標}

\textbf{年化報酬率:}策略以每次再平衡取預測分數最高的 top\_pct 股票做多,計算該截面平均報酬作為策略報酬,並依再平衡頻率換算年化報酬率。

\textbf{Sharpe Ratio:}Sharpe Ratio 定義為平均報酬除以波動後乘上年化因子,用於衡量風險調整後報酬。

\textbf{勝率:}勝率為正報酬期間比例,用於衡量策略穩定性。

\textbf{最大回撤:}程式中未計算最大回撤,若需納入,可額外補實作或引用後處理結果。【待補:是否加入最大回撤】

\subsection{基準比較}

\textbf{天真基準:}天真基準設定所有股票預測值為 0,計算 MSE/RMSE/MAE,並與模型結果比較改善幅度。

\textbf{台灣 50 ETF 基準:}使用 0050 報酬資料計算 forward-$k$ 日報酬,與策略累積報酬比較。

\subsection{評估流程}
流程為:
\begin{itemize}
    \item 載入 artifacts 與模型權重。
    \item 對訓練期與測試期分別計算預測指標。
    \item 產生 Daily IC、方向命中率、預測離散度、IC 分佈與累積報酬等視覺化報告。
    \item 若為 DMFM,另外輸出注意力權重分佈以進行解釋性分析。
\end{itemize}
