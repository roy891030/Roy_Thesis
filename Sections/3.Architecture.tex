% 請確保在 main.tex 或 preamble 中已載入以下套件:
% \usepackage{algorithm}
% \usepackage{algpseudocode}
% \usepackage{amsmath}
% \usepackage{booktabs}
% \usepackage{longtable} % 54 個變數表需要跨頁

\chapter{實驗設計}
\label{ch:experiment_design}

本章說明整體實驗流程,涵蓋資料來源與預處理、特徵工程與圖結構建構、模型架構與訓練設定、損失函數與評估指標。實驗目標在於檢驗 Deep Multi-Factor Model(DMFM)於台灣股票市場之預測表現,並與基準模型進行對照。

% =========================================================
\section{資料來源與預處理}
\label{sec:data}
% =========================================================

\subsection{資料來源}
\label{subsec:data_source}

資料取自 \textbf{TEJ 台灣經濟新報(Taiwan Economic Journal, TEJ)}之台灣股票市場日頻資料,並以 CSV 形式匯出以供後續處理與模型訓練使用。資料以「交易日\(\times\)股票」為觀測單位,至少包含日期(Date)、股票代碼(Stock ID)與收盤價(Close)等必要欄位;同時保留開高低收(OHLC)、成交量/成交值、市值、流通股數、估值指標(如 PB、PS)與產業分類等資訊,以支援特徵建置與圖結構構建。樣本期間由參數 \texttt{start\_date} 與 \texttt{end\_date} 控制,並以台股交易日序列作為時間索引基準。

除 TEJ 可直接取得之原始欄位外,模型輸入之技術特徵由價格與成交量等基礎序列計算而得;其中 RSI、MACD、KD 等指標主要透過 \texttt{TA-Lib} 套件生成,再彙整為固定的特徵集合供模型輸入。為避免於文字中重複列舉欄位與變數定義,TEJ 原始欄位彙整於表 \ref{tab:tej_raw_fields},最終模型輸入之 54 個特徵摘要見表 \ref{tab:feature_cols_54}(完整清單見附錄表 \ref{tab:feature_cols_54_full})。

\subsection{TEJ 資料欄位與研究變數表格}
\label{subsec:tej_variables}

所有輸入資料均由 TEJ 匯出為日頻 CSV。匯出檔案包含:(1)市場交易與公司屬性之原始欄位(表 \ref{tab:tej_raw_fields}),以及(2)由程式以 TEJ 原始欄位衍生之技術特徵集合 \texttt{feature\_cols (n=54)}(表 \ref{tab:feature_cols_54})。

\begin{table}[htbp]
\centering
\small
\caption{TEJ 匯出之原始欄位(資料來源欄位)}
\label{tab:tej_raw_fields}
\begin{tabular}{lll}
\toprule
\textbf{TEJ 欄位名稱} & \textbf{用途} & \textbf{備註(單位/說明)} \\
\midrule
年月日 & 時間索引 & 交易日(Date)\\
開盤價(元) & 交易價格 & Open(元)\\
最高價(元) & 交易價格 & High(元)\\
最低價(元) & 交易價格 & Low(元)\\
收盤價(元) & 交易價格 & Close(元)\\
成交量(千股) & 交易量能 & Volume(千股)\\
成交值(千元) & 交易金額 & Amount(千元)\\
本益比-TEJ & 估值資訊 & Price-to-Earnings(TEJ 口徑)\\
股價淨值比-TSE & 估值資訊 & Price-to-Book(TSE 口徑)\\
證券代碼\_純代碼 & 個股識別 & Stock ID\\
證券名稱 & 個股識別 & 公司名稱\\
TEJ產業\_名稱 & 產業資訊 & 產業分類(TEJ 分類;用於建構產業圖)\\
\bottomrule
\end{tabular}
\end{table}

\subsection{資料預處理}
\label{subsec:preprocess}

資料預處理主要涵蓋資料格式統一、特徵與標籤計算、以及張量化整理,並以一致規則處理缺失值以降低訓練偏誤。整體流程如演算法 \ref{alg:preprocessing} 所示。

\begin{algorithm}[htbp]
\caption{資料預處理流程}
\label{alg:preprocessing}
\begin{algorithmic}[1]
\State 讀取 TEJ 匯出之日頻資料(至少包含 Date、Stock ID、Close;其餘欄位供特徵與圖結構使用)
\State 依 Stock ID 與 Date 排序;進行日期格式與數值型態轉換,並做基本缺漏/異常檢核
\State 依價格與量能序列計算技術特徵(含 TA-Lib 指標),形成 \texttt{feature\_cols}
\State 計算標籤: $y_t = \frac{P_{t+k} - P_t}{P_t}$ \Comment{$k$ 日未來報酬率(forward return)}
\State 由產業欄位建構產業圖 $E_{ind}$;由股票清單建構全市場圖 $E_{uni}$
\State 建立特徵張量 $F^t \in \mathbb{R}^{T \times N \times F}$ 與標籤張量 $y^t \in \mathbb{R}^{T \times N}$
\State 特徵 NaN/Inf 以 0 取代;標籤 NaN 保留並於訓練/評估以 mask 排除
\State 特徵正規化由模型端 Batch Normalization(BatchNorm)處理
\end{algorithmic}
\end{algorithm}

\textbf{缺失值處理(Missing values):}不對缺失值進行插補,以避免引入主觀假設。特徵層面將 NaN/Inf 以 0 取代以維持張量完整性;標籤層面保留 NaN,並在訓練與評估時以有效樣本遮罩(mask)排除,使模型僅在可觀測標籤上學習。

\textbf{特徵標準化(Normalization):}資料前處理階段不額外做截面標準化,特徵保留原始尺度;模型端於輸入層使用 \textbf{Batch Normalization(BatchNorm)} 進行正規化:
\begin{equation}
x_{normalized} = \frac{x - \mu_{batch}}{\sqrt{\sigma^2_{batch} + \epsilon}}
\label{eq:batchnorm}
\end{equation}
其中 $\mu_{batch}$ 與 $\sigma^2_{batch}$ 為 mini-batch 統計量,$\epsilon$ 為數值穩定項(預設 $10^{-5}$)。

\textbf{標籤定義(Label definition):}標籤定義為未來 $k$ 日報酬率(forward-$k$ return):
\begin{equation}
y_t = \frac{P_{t+k} - P_t}{P_t}
\label{eq:return}
\end{equation}
其中 $P_t$ 為時間 $t$ 收盤價,$k$ 為預測視窗(預設 $k=5$)。

\subsection{資料集劃分}
\label{subsec:split}

採時間序列切分以避免資訊洩漏(look-ahead bias)。資料依時間順序切分為訓練集與測試集,比例為 80\%:20\%(前 80\% 為訓練、後 20\% 為測試)。另為檢驗不同樣本長度下之穩健性,設計短期/中期/長期三種資料長度,並皆採用相同的 80\%/20\% 時間切分:
\begin{itemize}
    \item \textbf{短期資料:}2019-09-16 $\sim$ 2020-12-31(共 319 個交易日;訓練 255 日、測試 64 日)
    \item \textbf{中期資料:}2019-09-16 $\sim$ 2022-12-31(共 809 個交易日;訓練 647 日、測試 162 日)
    \item \textbf{長期資料:}2019-09-16 $\sim$ 2025-09-12(共 1460 個交易日;訓練 1168 日、測試 292 日)
\end{itemize}

% =========================================================
\section{特徵工程與圖結構}
\label{sec:features}
% =========================================================

\subsection{特徵選取與分類}
\label{subsec:feature_groups}

模型輸入特徵涵蓋報酬、趨勢、波動、量能、流動性與技術指標等訊號來源,並以多視窗(rolling window)構建不同時間尺度之市場資訊。各特徵之\textbf{實際使用變數名稱、定義、視窗長度與對應 TEJ 原始欄位}已於附錄表 \ref{tab:feature_cols_54_full} 完整列示。除 TEJ 匯出欄位外,技術指標部分由 \texttt{TA-Lib} 依價格與量能序列計算,不另引入外部第三方因子資料。

\renewcommand{\arraystretch}{1.15}
\begin{table}[htbp]
\centering
\caption{模型輸入特徵摘要(完整 54 項清單見附錄表 \ref{tab:feature_cols_54_full})}
\label{tab:feature_cols_54}
\begin{tabular}{p{3.1cm} p{3.0cm} p{6.8cm} p{1.6cm}}
\toprule
\textbf{類別} & \textbf{代表特徵} & \textbf{定義(由 TEJ 欄位計算)} & \textbf{視窗} \\
\midrule
報酬 & ret & $P_t/P_{t-k}-1$(TEJ 收盤價) & 1/5/20 \\
反轉 & rev & $-\texttt{ret}_k$(TEJ 收盤價) & 1/5/10 \\
趨勢 & px\_over\_sma & $P_t/\text{SMA}_k(P)_t-1$(TEJ 收盤價) & 5/20/60 \\
動能差 & mom\_diff & $\texttt{ret}_{k_1}-\texttt{ret}_{k_2}$(例如 $10-1$, $20-1$;TEJ 收盤價) & -- \\
波動 & std\_ret & $\text{Std}(r_{t-k+1},\ldots,r_t)$(TEJ 收盤價) & 5/20/60 \\
分配 & skew, kurt & 20 日報酬偏度/峰度(TEJ 收盤價) & 20 \\
量能 & vol\_over\_ma & $V_t/\text{SMA}_k(V)_t-1$(TEJ 成交量) & 5/20 \\
量能 & up\_vol, down\_vol & 上/下漲日平均量(TEJ 成交量;以日報酬正負分組) & 20 \\
流動性 & amihud & $\frac{1}{k}\sum \frac{|r|}{\text{Amt}+\varepsilon}$(TEJ 成交值) & 5/20 \\
技術指標 & rsi, macd, stoch\_k & RSI / MACD / KD(以 TEJ 序列由 TA-Lib 計算) & 14/20 \\
風險 & atr, mdd & ATR(TEJ 高低收)/ 最大回撤(TEJ 收盤價) & 14/20 \\
風險 & beta, idio\_vol & rolling beta / 特質波動(以個股與市場報酬 rolling 估計) & 60 \\
區間極值 & roll\_max, roll\_min & rolling max/min(TEJ 收盤價) & 5/10/20/60 \\
區間位置 & pct\_pos & $\frac{P_t-\min(P)}{\max(P)-\min(P)+\varepsilon}$(TEJ 收盤價) & 5/10/20/60 \\
位置差距 & pct\_to\_high, pct\_to\_low & 距高/低點比例(以區間 $\max/\min$ 與 $P_t$ 計;TEJ 收盤價) & 20 \\
標準化 & zscore\_close & $\frac{P_t-\mu_k}{\sigma_k+\varepsilon}$(TEJ 收盤價) & 20/60 \\
\bottomrule
\end{tabular}
\end{table}
\renewcommand{\arraystretch}{1.0}

\textbf{產業分類(Industry classification):}產業欄位採用 TEJ 提供之 \texttt{TEJ產業\_名稱}。該分類於樣本期間內具相對穩定性,適合用於建構產業層級圖結構;實驗中不額外進行主觀重分群。

\subsection{圖結構設計與建構流程}
\label{subsec:graph_build}

\textbf{產業圖(Industry Graph):}以「同產業股票完全連結」為原則,將同一產業內之股票視為相互連結,並為每檔股票加入自環(self-loop),以捕捉產業內共通訊號。

\textbf{全市場圖(Universe Graph):}採完全圖(complete graph)設計,所有股票相互連結並包含自環,用以表徵跨產業之市場共同因子影響。

\textbf{表示方式(Representation):}圖結構以 PyTorch Geometric(PyG)之 \texttt{edge\_index} 表示(形狀為 $[2, E]$)。實作上先依規則建立連結關係,再轉換為 \texttt{edge\_index} 以供 GAT 層運算。

\textbf{建構流程(Construction):}由輸入 CSV 讀取產業欄位並依股票代碼分組建立產業圖;若產業欄缺漏,則以單一產業處理。全市場圖依股票清單建立完全連結。圖結構與特徵張量共同保存為 artifacts,以支援訓練與評估之重複使用。

% =========================================================
\section{模型設計}
\label{sec:models}
% =========================================================

\subsection{基準模型:GATRegressor}
\label{subsec:lstm_baseline}

基準模型採用 \textbf{GATRegressor} 作為對照:以兩層 Graph Attention Network(GAT)於產業圖上進行訊息傳遞,並以線性層輸出預測值。相較於 DMFM,GATRegressor 不包含「產業中性化+全市場中性化」之階層式結構,亦不含因子注意力模組,因此可作為較簡化之圖模型基準。超參數設定盡量與 DMFM 對齊(例如 hidden dimension、heads、dropout),以提升可比性。

%(若後續補上 LSTM baseline,可在此節加一小段描述,或另設小節。)

\subsection{DMFM 模型架構(Wei et al. 2022)}
\label{subsec:dmfm_model}

DMFM 之核心流程可概括為:「特徵編碼 $\rightarrow$ 產業中性化 $\rightarrow$ 全市場中性化 $\rightarrow$ 階層式拼接 $\rightarrow$ 深度因子學習 $\rightarrow$ 因子注意力」。模型在每個交易日 $t$ 以截面方式處理所有股票之特徵,並輸出每檔股票之預測因子值。

\textbf{步驟 1:特徵編碼(Feature encoder):}原始特徵 $F^t$ 經 BatchNorm 後輸入 MLP,映射為 $C^t$(維度為 \texttt{hidden\_dim}),作為股票之語境表示。

\textbf{步驟 2:產業中性化(Industry neutralization):}以產業圖進行 GAT 運算得到產業影響 $H_I^t$,並定義產業中性特徵為
$C_I^t = C^t - H_I^t$。

\textbf{步驟 3:全市場中性化(Universe neutralization):}以全市場圖在 $C_I^t$ 上進行 GAT 運算得到 $H_U^t$,並定義全市場中性特徵為
$C_U^t = C_I^t - H_U^t$。

\textbf{步驟 4:階層式特徵拼接(Hierarchical concatenation):}拼接三層表示形成
$H^t = [C^t \, || \, C_I^t \, || \, C_U^t]$,
以同時保留原始語境與兩層中性化資訊。

\textbf{步驟 5:深度因子學習(Deep factor learning):}將 $H^t$ 輸入 decoder MLP,輸出深度因子 $f^t$(每檔股票一個預測值)。

\textbf{步驟 6:因子注意力(Factor attention):}模型同時學習特徵注意力權重 $a^t$,形成 $\hat{f}^t = F^t \odot a^t$,用以估計驅動因子輸出之重要特徵,並作為監督與解釋性分析依據。

\textbf{超參數配置(Hyperparameters):}預設為 \texttt{hidden\_dim}=64、\texttt{heads}=2、\texttt{dropout}=0.1、\texttt{epochs}=200、\texttt{learning rate}=1e$-$4、\texttt{weight decay}=0.01、$\lambda_{attn}=0.1$、$\lambda_{IC}=1.0$、\texttt{patience}=30。

% =========================================================
\section{損失函數}
\label{sec:loss}
% =========================================================

\subsection{總損失函數}
採用 DMFM 對應之損失函數:
\begin{equation}
L = \lambda_{attn} \cdot \lVert f - \hat{f} \rVert + \lambda_{IC} \cdot (1 - IC) - \lambda_b \cdot b
\label{eq:loss}
\end{equation}
其中 $\lVert f - \hat{f} \rVert$ 為注意力估計誤差,$IC$ 為資訊係數(Information Coefficient, IC),$b$ 為截面回歸得到的因子收益項。實作中 $\lambda_b$ 固定為 0.01,以降低因子收益項之不穩定性。

\textbf{IC(Information Coefficient):}以 Pearson correlation 計算,基於每個交易日之截面預測與真實 forward-$k$ 日報酬,衡量預測排序有效性。

\textbf{優化器(Optimizer):}使用 AdamW,並加入 weight decay 以抑制過度擬合。

% =========================================================
\section{評估指標}
\label{sec:metrics}
% =========================================================

\subsection{預測品質指標}
\textbf{IC / Daily IC / ICIR:}IC 為整體樣本之 Pearson correlation;Daily IC 為逐交易日 IC 之平均;ICIR(Information Coefficient Information Ratio, ICIR)定義為 Daily IC 平均除以其標準差,用以衡量穩健性。

\textbf{產業中性 IC(Industry-neutral IC):}為排除產業共同波動之影響,先在每交易日、每產業內分別對預測與真實報酬去均值(industry de-meaning),再對殘差計算 Pearson correlation。令 $\hat{y}_{i,t}$ 與 $y_{i,t}$ 分別為股票 $i$ 在日 $t$ 的預測與真實報酬,$g(i)$ 表示產業別,則
\begin{align}
\tilde{\hat{y}}_{i,t} &= \hat{y}_{i,t} - \frac{1}{|G_{g(i),t}|}\sum_{j \in G_{g(i),t}} \hat{y}_{j,t}, \\
\tilde{y}_{i,t} &= y_{i,t} - \frac{1}{|G_{g(i),t}|}\sum_{j \in G_{g(i),t}} y_{j,t},
\end{align}
其中 $G_{g(i),t}$ 為日 $t$ 時產業 $g(i)$ 的有效股票集合(排除標籤為 NaN 者)。產業中性 IC 定義為
\begin{equation}
IC^{\text{ind-neutral}}_t = \text{Corr}\left(\tilde{\hat{y}}_{\cdot,t}, \tilde{y}_{\cdot,t}\right).
\end{equation}

\textbf{誤差指標(Error metrics):}以 MSE、RMSE 與 MAE 衡量預測誤差。

\subsection{投資組合績效指標}
\textbf{年化報酬率(Annualized return):}以每次再平衡選取預測分數最高之 \texttt{top\_pct} 股票等權做多,計算截面平均報酬作為策略報酬,並依再平衡頻率年化。

\textbf{Sharpe Ratio:}以平均報酬除以波動後再乘年化因子,衡量風險調整後報酬。

\textbf{勝率(Hit ratio):}正報酬期間比例,用以衡量策略穩定性。

\subsection{基準比較與流程摘要}
\textbf{天真基準(Naive baseline):}令所有股票預測值為 0,計算 MSE/RMSE/MAE 作為誤差下界參考。

\textbf{台灣 50 ETF 基準:}以 0050 之 forward-$k$ 日報酬作為市場基準,與策略累積報酬比較。

評估流程為:載入 artifacts 與模型權重後,分別於訓練期與測試期計算預測與投資組合指標;並輸出 Daily IC、IC 分佈與累積報酬等視覺化結果,以利比較不同模型之表現。對於 DMFM,另輸出注意力權重分佈作為解釋性分析依據。
