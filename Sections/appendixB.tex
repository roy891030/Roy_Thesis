\chapter{詳細模型設計程式碼}
\label{appendix:code}

% 附錄 B:放置關鍵的程式碼片段

\section{LSTM 模型實作}
\label{appendix:lstm_code}

% 在此放置 LSTM 模型的關鍵程式碼

以下為 LSTM 模型的 PyTorch 實作範例:

\begin{lstlisting}[language=Python, caption=LSTM 模型架構]
import torch
import torch.nn as nn

class LSTMModel(nn.Module):
    def __init__(self, input_dim, hidden_dim, num_layers):
        super(LSTMModel, self).__init__()
        self.lstm = nn.LSTM(input_dim, hidden_dim,
                           num_layers, batch_first=True)
        self.fc = nn.Linear(hidden_dim, 1)

    def forward(self, x):
        lstm_out, _ = self.lstm(x)
        predictions = self.fc(lstm_out[:, -1, :])
        return predictions
\end{lstlisting}

(可根據實際程式碼調整)

\section{GAT 模型實作}
\label{appendix:gat_code}

% 在此放置 GAT 模型的關鍵程式碼

以下為 GAT 模型的關鍵實作:

\begin{lstlisting}[language=Python, caption=GAT 模型架構]
import torch
import torch.nn as nn
import torch.nn.functional as F

class GATLayer(nn.Module):
    def __init__(self, in_features, out_features, heads=8):
        super(GATLayer, self).__init__()
        # GAT implementation
        # (根據實際實作填寫)
        pass

    def forward(self, x, adj):
        # Attention mechanism
        # (根據實際實作填寫)
        pass
\end{lstlisting}

(可根據實際程式碼調整)

\section{DMFM 模型實作}
\label{appendix:dmfm_code}

% 在此放置 DMFM 模型的關鍵程式碼

以下為 DMFM 提案模型的實作:

\begin{lstlisting}[language=Python, caption=DMFM 模型架構]
# DMFM model implementation
# (根據實際實作填寫)
\end{lstlisting}

(可根據實際程式碼調整)

\section{資料預處理程式碼}
\label{appendix:preprocessing_code}

% 在此放置資料預處理的關鍵程式碼

\begin{lstlisting}[language=Python, caption=資料預處理流程]
import pandas as pd
import numpy as np

def preprocess_data(df):
    # Data preprocessing steps
    # (根據實際實作填寫)
    pass
\end{lstlisting}

(可根據實際程式碼調整)
