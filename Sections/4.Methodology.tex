\chapter{實驗結果}
\label{ch:results}

% 本章呈現實驗結果,包含模型對比、投組績效與特徵重要性分析

\section{模型對比}
\label{sec:model_comparison}

% 在此撰寫:
% - LSTM vs GAT vs DMFM 預測準確度比較
% - 各模型的 RMSE, MAE, R² 等指標
% - 不同市場條件下的表現差異
% - 結果分析與討論

(此處撰寫模型對比內容)

% 範例:可以插入比較表格
% \begin{table}[h]
% \centering
% \caption{模型預測準確度比較}
% \begin{tabular}{c|c|c|c}
% \toprule[1.25pt]
% 模型 & RMSE & MAE & R² \\
% \midrule
% LSTM & 0.025 & 0.018 & 0.65 \\
% GAT & 0.022 & 0.016 & 0.71 \\
% DMFM & \textbf{0.019} & \textbf{0.014} & \textbf{0.76} \\
% \bottomrule[1.25pt]
% \end{tabular}
% \label{table:model_comparison}
% \end{table}

\section{投組績效驗證}
\label{sec:portfolio_performance}

% 在此撰寫:
% - 回測期間與設定
% - 年化報酬率
% - Sharpe Ratio
% - 最大回撤
% - 與基準指數的比較(例如大盤指數)
% - 績效歸因分析

(此處撰寫投組績效驗證內容)

% 範例:可以插入績效圖表
% \begin{figure}[htb]
%   \centering
%   \includegraphics[width=0.9\textwidth]{img/portfolio_performance.png}
%   \caption{投資組合累積報酬曲線}
%   \label{fig:portfolio_performance}
% \end{figure}

\section{特徵重要性排名}
\label{sec:feature_importance}

% 在此撰寫:
% - 使用的特徵重要性分析方法
% - 技術指標重要性排名
% - 圖結構對預測的影響
% - 注意力權重分析(GAT/DMFM)
% - 結果解釋與討論

(此處撰寫特徵重要性排名內容)

% 範例:可以插入特徵重要性表格或圖表
% \begin{table}[h]
% \centering
% \caption{前十大重要特徵}
% \begin{tabular}{c|l|c}
% \toprule[1.25pt]
% 排名 & 特徵名稱 & 重要性分數 \\
% \midrule
% 1 & RSI\_14 & 0.156 \\
% 2 & MACD & 0.142 \\
% 3 & Volume\_MA\_20 & 0.128 \\
% ... & ... & ... \\
% \bottomrule[1.25pt]
% \end{tabular}
% \label{table:feature_importance}
% \end{table}
