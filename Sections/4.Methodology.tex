\chapter{實驗結果}
\label{ch:results}

本章整理各模型在相同資料切分下的測試結果,從(1)預測能力、(2)投組績效、(3)DMFM 特徵權重三個面向進行比較。

\section{模型對比}
\label{sec:model_comparison}

本節比較五個模型(Linear、XGBoost、LSTM、GAT、DMFM)在三種時間窗(Short/Medium/Long)之測試表現。主要指標如下:

\begin{itemize}
  \item \textbf{RMSE}:$\sqrt{\frac{1}{M}\sum_{t,i}(\hat{y}_{i,t}-y_{i,t})^2}$。
  \item \textbf{MAE}:$\frac{1}{M}\sum_{t,i}\left|\hat{y}_{i,t}-y_{i,t}\right|$。
  \item \textbf{IC}:以每日截面相關係數 $IC_t=\mathrm{corr}(\hat{y}_{:,t},y_{:,t})$ 計算,再彙整其平均值。
  \item \textbf{ICIR}:$\frac{\mathrm{mean}(IC_t)}{\mathrm{std}(IC_t)}$,衡量排序訊號的穩定性。
  \item \textbf{DirAcc}:$\frac{1}{M}\sum_{t,i}\mathbf{1}\!\left[\mathrm{sign}(\hat{y}_{i,t})=\mathrm{sign}(y_{i,t})\right]$。
\end{itemize}

其中 $M$ 為測試集中「交易日$\times$股票」樣本總數,$\hat{y}_{i,t}$ 與 $y_{i,t}$ 分別為模型預測值與實際報酬。

\begin{table}[H]
\centering
\caption{各模型於不同時間窗之預測指標比較(測試集)}
\label{tab:results_pred_metrics}
\begin{adjustbox}{max width=\textwidth}
\begin{tabular}{llrrrrr}
\toprule
時間窗 & 模型 & RMSE & MAE & IC & ICIR & DirAcc \\
\midrule
Short & Linear & \textbf{0.0555} & 0.0340 & 0.0266 & 0.4313 & 0.5225 \\
 & XGBoost & 0.0559 & 0.0344 & 0.0644 & \textbf{0.9551} & 0.5269 \\
 & LSTM & 0.0661 & 0.0394 & 0.0181 & 0.3267 & 0.5562 \\
 & GAT & 0.1206 & 0.1055 & -0.0225 & -0.2615 & 0.4169 \\
 & DMFM & 0.0563 & \textbf{0.0338} & \textbf{0.0721} & 0.8228 & \textbf{0.5821} \\
\midrule
Medium & Linear & \textbf{0.0502} & \textbf{0.0334} & 0.0249 & 0.2192 & 0.5092 \\
 & XGBoost & 0.0503 & 0.0335 & 0.0688 & 0.5968 & 0.5232 \\
 & LSTM & 0.0564 & 0.0367 & 0.0519 & 0.4525 & 0.5239 \\
 & GAT & 0.1293 & 0.1171 & -0.0013 & -0.0107 & 0.4504 \\
 & DMFM & 0.0627 & 0.0461 & \textbf{0.0815} & \textbf{0.6762} & \textbf{0.5565} \\
\midrule
Long & Linear & 0.0557 & 0.0359 & 0.0414 & 0.3865 & 0.5096 \\
 & XGBoost & \textbf{0.0556} & \textbf{0.0358} & \textbf{0.0837} & 0.7087 & 0.5271 \\
 & LSTM & 0.0604 & 0.0377 & 0.0344 & 0.3874 & 0.5350 \\
 & GAT & 0.0808 & 0.0605 & -0.0096 & -0.1032 & 0.5266 \\
 & DMFM & 0.1213 & 0.0914 & 0.0834 & \textbf{0.7854} & \textbf{0.5663} \\
\bottomrule
\end{tabular}
\end{adjustbox}
\end{table}


由表 \ref{tab:results_pred_metrics} 得到的主要觀察如下:
\begin{itemize}
  \item \textbf{排序能力}:XGBoost 在 Short/Medium/Long 三個時間窗皆有最高 IC 與 ICIR,代表排序訊號最穩定。
  \item \textbf{誤差表現}:Short、Medium 以 Linear 的 RMSE/MAE 最低;Long 則由 XGBoost 略優。
  \item \textbf{方向準確率與排序品質分離}:GAT 在 Short 的 DirAcc 較高,但 IC/ICIR 偏弱,顯示方向命中率高不必然代表排序訊號品質佳。
\end{itemize}

\begin{figure}[htbp]
  \centering
  \includegraphics[width=0.90\textwidth]{img/results_model_comparison_metrics.png}
  \caption{各模型在不同時間窗之關鍵預測指標比較(ICIR、IC、RMSE、DirAcc)。}
  \label{fig:results_model_comparison_metrics}
\end{figure}

\section{投組績效驗證}
\label{sec:portfolio_performance}

本節比較 GAT、DMFM 與基準組合在不同時間窗的年化報酬、Sharpe 與勝率。

\begin{table}[htbp]
\centering
\caption{GAT、DMFM 與基準於回測區間之績效比較}
\label{tab:results_portfolio_metrics}
\begin{adjustbox}{max width=0.92\textwidth}
\begin{tabular}{llrrrr}
\toprule
時間窗 & 組合 & 對齊期數($n$) & 年化報酬 & Sharpe & 勝率 \\
\midrule
Short & Benchmark & 13 & \textbf{0.7584} & \textbf{4.201} & \textbf{0.615} \\
 & GAT & 13 & -0.1206 & -2.526 & 0.462 \\
 & DMFM & 13 & 0.1085 & 2.929 & 0.538 \\
\midrule
Medium & Benchmark & 33 & -0.0890 & \textbf{-0.341} & 0.424 \\
 & GAT & 33 & \textbf{-0.0746} & -0.501 & \textbf{0.515} \\
 & DMFM & 33 & -0.1611 & -2.341 & 0.364 \\
\midrule
Long & Benchmark & 35 & -0.0335 & -0.113 & 0.486 \\
 & GAT & 35 & \textbf{0.0566} & \textbf{0.816} & \textbf{0.514} \\
 & DMFM & 35 & -0.0286 & -0.330 & 0.514 \\
\bottomrule
\end{tabular}
\end{adjustbox}
\end{table}


由表 \ref{tab:results_portfolio_metrics} 的比較可得:
\begin{itemize}
  \item \textbf{Short}:Benchmark 最佳;DMFM 為正報酬但仍落後基準,GAT 為負報酬。
  \item \textbf{Medium}:三者年化皆為負,且 DMFM 的 Sharpe 最弱。
  \item \textbf{Long}:GAT 取得最佳年化(0.0566)與 Sharpe(0.816),顯著優於同期基準。
\end{itemize}

\begin{figure}[htbp]
  \centering
  \includegraphics[width=\textwidth]{img/results_cum_returns_grid.png}
  \caption{GAT 與 DMFM 在 Short/Medium/Long 三種時間窗之累積報酬曲線比較。}
  \label{fig:results_cum_returns_grid}
\end{figure}

\section{特徵重要性排名}
\label{sec:feature_importance}

為了解 DMFM 在不同時間窗下偏好的訊號來源,本節依注意力權重統計前五特徵。表 \ref{tab:results_dmfm_attention} 列出各時間窗的 Top-5 結果。

\begin{table}[htbp]
\centering
\caption{DMFM 注意力權重前五特徵(依時間窗)}
\label{tab:results_dmfm_attention}
\begin{adjustbox}{max width=\textwidth}
\begin{tabular}{lccccc}
\toprule
時間窗 & Rank1 & Rank2 & Rank3 & Rank4 & Rank5 \\
\midrule
Short & \texttt{amihud\_5} (0.0278) & \texttt{rev\_1} (0.0225) & \texttt{up\_vol\_20} (0.0205) & \texttt{roll\_min\_5} (0.0196) & \texttt{roll\_max\_10} (0.0196) \\
Medium & \texttt{ps} (0.0608) & \texttt{roll\_max\_10} (0.0370) & \texttt{amihud\_5} (0.0318) & \texttt{down\_vol\_20} (0.0302) & \texttt{rev\_10} (0.0302) \\
Long & \texttt{ps} (0.0274) & \texttt{amihud\_20} (0.0246) & \texttt{roll\_min\_5} (0.0226) & \texttt{amihud\_5} (0.0220) & \texttt{roll\_max\_10} (0.0220) \\
\bottomrule
\end{tabular}
\end{adjustbox}
\end{table}


由表 \ref{tab:results_dmfm_attention} 可得兩項重點:
\begin{itemize}
  \item \texttt{amihud\_5} 與 \texttt{roll\_max\_10} 在多個時間窗重複出現,代表短期流動性與區間位置特徵具有持續影響力。
  \item \texttt{ps} 在 Medium/Long 皆為第一,顯示中長期設定下估值訊號的重要性提升。
\end{itemize}

\begin{figure}[htbp]
  \centering
  \includegraphics[width=\textwidth]{img/results_dmfm_attention_top5.png}
  \caption{DMFM 在不同時間窗的注意力前五特徵權重。}
  \label{fig:results_dmfm_attention_top5}
\end{figure}
