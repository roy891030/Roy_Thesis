\chapter{實驗結果}
\label{ch:results}

本章整理各模型在相同資料切分下的測試結果,從(1)預測能力、(2)投組績效、(3)DMFM 特徵權重三個面向進行比較。

\section{模型對比}
\label{sec:model_comparison}

\subsection{評估指標說明}

本研究比較五個模型(Linear、XGBoost、LSTM、GAT、DMFM)在三種時間窗(Short/Medium/Long)之測試表現。指標設計分成兩類:RMSE/MAE 用於衡量點預測誤差,IC/ICIR/DirAcc 用於衡量排序與方向判斷是否可用於選股決策。

\begin{itemize}
  \item \textbf{RMSE}:$\sqrt{\frac{1}{M}\sum_{t,i}(\hat{y}_{i,t}-y_{i,t})^2}$。
  \item \textbf{MAE}:$\frac{1}{M}\sum_{t,i}\left|\hat{y}_{i,t}-y_{i,t}\right|$。
  \item \textbf{IC}:以每日截面相關係數 $IC_t=\mathrm{corr}(\hat{y}_{:,t},y_{:,t})$ 計算,再彙整其平均值。
  \item \textbf{ICIR}:$\frac{\mathrm{mean}(IC_t)}{\mathrm{std}(IC_t)}$,衡量排序訊號的穩定性。
  \item \textbf{DirAcc}:$\frac{1}{M}\sum_{t,i}\mathbf{1}\!\left[\mathrm{sign}(\hat{y}_{i,t})=\mathrm{sign}(y_{i,t})\right]$。
\end{itemize}

其中 $M$ 為測試集中「交易日$\times$股票」樣本總數,$\hat{y}_{i,t}$ 與 $y_{i,t}$ 分別為模型預測值與實際報酬。

\subsection{模型比較結果}

\begin{samepage}
\begin{table}[H]
\centering
\caption{各模型於不同時間窗之預測指標比較(測試集)}
\label{tab:results_pred_metrics}
\begin{adjustbox}{max width=\textwidth}
\begin{tabular}{llrrrrr}
\toprule
時間窗 & 模型 & RMSE & MAE & IC & ICIR & DirAcc \\
\midrule
Short & Linear & \textbf{0.0555} & 0.0340 & 0.0266 & 0.4313 & 0.5225 \\
 & XGBoost & 0.0559 & 0.0344 & 0.0644 & \textbf{0.9551} & 0.5269 \\
 & LSTM & 0.0661 & 0.0394 & 0.0181 & 0.3267 & 0.5562 \\
 & GAT & 0.1206 & 0.1055 & -0.0225 & -0.2615 & 0.4169 \\
 & DMFM & 0.0563 & \textbf{0.0338} & \textbf{0.0721} & 0.8228 & \textbf{0.5821} \\
\midrule
Medium & Linear & \textbf{0.0502} & \textbf{0.0334} & 0.0249 & 0.2192 & 0.5092 \\
 & XGBoost & 0.0503 & 0.0335 & 0.0688 & 0.5968 & 0.5232 \\
 & LSTM & 0.0564 & 0.0367 & 0.0519 & 0.4525 & 0.5239 \\
 & GAT & 0.1293 & 0.1171 & -0.0013 & -0.0107 & 0.4504 \\
 & DMFM & 0.0627 & 0.0461 & \textbf{0.0815} & \textbf{0.6762} & \textbf{0.5565} \\
\midrule
Long & Linear & 0.0557 & 0.0359 & 0.0414 & 0.3865 & 0.5096 \\
 & XGBoost & \textbf{0.0556} & \textbf{0.0358} & \textbf{0.0837} & 0.7087 & 0.5271 \\
 & LSTM & 0.0604 & 0.0377 & 0.0344 & 0.3874 & 0.5350 \\
 & GAT & 0.0808 & 0.0605 & -0.0096 & -0.1032 & 0.5266 \\
 & DMFM & 0.1213 & 0.0914 & 0.0834 & \textbf{0.7854} & \textbf{0.5663} \\
\bottomrule
\end{tabular}
\end{adjustbox}
\end{table}


\begin{figure}[H]
  \centering
  \includegraphics[width=0.86\textwidth]{img/results_model_comparison_metrics.png}
  \caption{各模型在不同時間窗之關鍵預測指標比較(ICIR、IC、RMSE、DirAcc)。}
  \label{fig:results_model_comparison_metrics}
\end{figure}
表 \ref{tab:results_pred_metrics} 先呈現完整數值,圖 \ref{fig:results_model_comparison_metrics} 進一步以視覺化方式呈現同一組比較結果,便於跨模型與跨時間窗對照。
\end{samepage}

由表 \ref{tab:results_pred_metrics} 與圖 \ref{fig:results_model_comparison_metrics} 可得:
\begin{itemize}
  \item \textbf{誤差表現}:點預測誤差仍以 Linear/XGBoost 較低;DMFM 與 GAT 的 RMSE/MAE 較高,顯示排序訊號優勢不必然伴隨最小化誤差。
  \item \textbf{排序能力}:DMFM 在 Short/Medium 的 IC 與 DirAcc 皆為最佳;在 Long 視窗中,DMFM 與 XGBoost 的 IC 接近(0.0834 vs 0.0837),且 DMFM 的 ICIR 最高(0.7854)。
  \item \textbf{模型定位差異}:DMFM 在排序相關指標與方向準確率上有明顯提升,而 Linear/XGBoost 仍在誤差指標上保有優勢。
\end{itemize}
其餘預測輸出(含 MSE、Daily IC 與完整模型彙整)見附錄 \ref{sec:appendix_extra_outputs}。
\FloatBarrier

\section{投組績效驗證}
\label{sec:portfolio_performance}

本節比較 GAT、DMFM 與 0050 在不同時間窗的年化報酬、Sharpe 與勝率。回測選股流程為:在每次再平衡日,先依模型預測分數由高至低排序,再選取前 \texttt{top\_pct} 股票等權做多,持有至下一次再平衡。表 \ref{tab:results_portfolio_metrics} 彙整績效指標;圖 \ref{fig:results_cum_returns_grid} 對應呈現 Short/Medium/Long 三種時間窗的累積報酬曲線比較(已收錄於圖目錄)。
其中「對齊的天數」表示策略與 0050 在同一時間軸可直接比較的對齊後樣本天數。

\begin{table}[htbp]
\centering
\caption{GAT、DMFM 與基準於回測區間之績效比較}
\label{tab:results_portfolio_metrics}
\begin{adjustbox}{max width=0.92\textwidth}
\begin{tabular}{llrrrr}
\toprule
時間窗 & 組合 & 對齊期數($n$) & 年化報酬 & Sharpe & 勝率 \\
\midrule
Short & Benchmark & 13 & \textbf{0.7584} & \textbf{4.201} & \textbf{0.615} \\
 & GAT & 13 & -0.1206 & -2.526 & 0.462 \\
 & DMFM & 13 & 0.1085 & 2.929 & 0.538 \\
\midrule
Medium & Benchmark & 33 & -0.0890 & \textbf{-0.341} & 0.424 \\
 & GAT & 33 & \textbf{-0.0746} & -0.501 & \textbf{0.515} \\
 & DMFM & 33 & -0.1611 & -2.341 & 0.364 \\
\midrule
Long & Benchmark & 35 & -0.0335 & -0.113 & 0.486 \\
 & GAT & 35 & \textbf{0.0566} & \textbf{0.816} & \textbf{0.514} \\
 & DMFM & 35 & -0.0286 & -0.330 & 0.514 \\
\bottomrule
\end{tabular}
\end{adjustbox}
\end{table}


\begin{figure}[H]
  \centering
  \includegraphics[width=\textwidth]{img/results_cum_returns_grid.png}
  \caption{GAT 與 DMFM 在 Short/Medium/Long 三種時間窗之累積報酬曲線比較。}
  \label{fig:results_cum_returns_grid}
\end{figure}

由表 \ref{tab:results_portfolio_metrics} 與圖 \ref{fig:results_cum_returns_grid} 搭配比較可得:
\begin{itemize}
  \item \textbf{Short}:DMFM 年化 0.9298、Sharpe 8.002、勝率 0.846,優於 GAT 與 0050。
  \item \textbf{Medium}:DMFM(年化 0.3734、Sharpe 1.274)優於 GAT(0.1158、0.570)與 0050(負年化)。
  \item \textbf{Long}:DMFM 仍維持最佳(年化 0.3131、Sharpe 1.369),GAT 在長窗轉為負年化。
\end{itemize}
完整累積報酬圖(含各時間窗、各模型)見附錄 \ref{sec:appendix_extra_outputs};其餘回測輸出(Daily IC、IC 分佈、月度命中率等)亦整理於附錄。
\FloatBarrier

\section{特徵重要性排名}
\label{sec:feature_importance}

為了解 DMFM 在不同時間窗下偏好的訊號來源,本節依注意力權重統計前五特徵。表 \ref{tab:results_dmfm_attention} 列出各時間窗的 Top-5 結果,圖 \ref{fig:results_dmfm_attention_top5} 以圖形方式對照相同資訊。

\begin{table}[htbp]
\centering
\caption{DMFM 注意力權重前五特徵(依時間窗)}
\label{tab:results_dmfm_attention}
\begin{adjustbox}{max width=\textwidth}
\begin{tabular}{lccccc}
\toprule
時間窗 & Rank1 & Rank2 & Rank3 & Rank4 & Rank5 \\
\midrule
Short & \texttt{amihud\_5} (0.0278) & \texttt{rev\_1} (0.0225) & \texttt{up\_vol\_20} (0.0205) & \texttt{roll\_min\_5} (0.0196) & \texttt{roll\_max\_10} (0.0196) \\
Medium & \texttt{ps} (0.0608) & \texttt{roll\_max\_10} (0.0370) & \texttt{amihud\_5} (0.0318) & \texttt{down\_vol\_20} (0.0302) & \texttt{rev\_10} (0.0302) \\
Long & \texttt{ps} (0.0274) & \texttt{amihud\_20} (0.0246) & \texttt{roll\_min\_5} (0.0226) & \texttt{amihud\_5} (0.0220) & \texttt{roll\_max\_10} (0.0220) \\
\bottomrule
\end{tabular}
\end{adjustbox}
\end{table}


\begin{figure}[H]
  \centering
  \includegraphics[width=\textwidth]{img/results_dmfm_attention_top5.png}
  \caption{DMFM 在不同時間窗的注意力前五特徵權重。}
  \label{fig:results_dmfm_attention_top5}
\end{figure}

由表 \ref{tab:results_dmfm_attention} 與圖 \ref{fig:results_dmfm_attention_top5} 可得兩項重點:
\begin{itemize}
  \item 區間位置與量價特徵在多個時間窗持續出現(如 \texttt{roll\_max\_5}、\texttt{pct\_to\_low\_20}、\texttt{down\_vol\_20}),顯示 DMFM 偏好結合價格區間與量能訊號。
  \item 流動性與動能訊號仍具影響力(如 \texttt{amihud\_20}、\texttt{amihud\_5}、\texttt{ret\_20}、\texttt{macd\_signal}),反映模型在不同時間窗會切換關注的主導因子。
\end{itemize}
更多注意力與輔助輸出亦整理於附錄 \ref{sec:appendix_extra_outputs}。
\FloatBarrier
