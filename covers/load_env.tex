% 此檔案被 main.tex 引入,不需要 \documentclass
% 字體與套件設定

\usepackage[no-math]{fontspec}   %加這個就可以設定字體 

\usepackage{xeCJK}       %讓中英文字體分開設置
\usepackage[UTF8,fontset=none,scheme=plain]{ctex}
% 可攜式字型設定:優先使用常見 macOS 字型,找不到時自動 fallback
\IfFontExistsTF{Times New Roman}
  {\setmainfont{Times New Roman}}
  {\setmainfont{TeX Gyre Termes}}

\IfFontExistsTF{Kaiti TC}
  {\setCJKmainfont[AutoFakeBold={4},AutoFakeSlant=0.2]{Kaiti TC}}
  {\IfFontExistsTF{Songti TC}
    {\setCJKmainfont[AutoFakeBold={4},AutoFakeSlant=0.2]{Songti TC}}
    {\setCJKmainfont[AutoFakeBold={4},AutoFakeSlant=0.2]{PingFang TC}}}

\IfFontExistsTF{PingFang TC}
  {\setCJKsansfont{PingFang TC}}
  {\setCJKsansfont{Songti TC}}

\IfFontExistsTF{Menlo}
  {\setCJKmonofont{Menlo}}
  {\setCJKmonofont{Courier New}}

\usepackage{CJKnumb}

\usepackage{subcaption}
\usepackage{caption}

\usepackage{indentfirst}
\usepackage{array}
\usepackage{multirow}
\usepackage{amsthm}
\usepackage{amsmath}
\usepackage{amsfonts}
\usepackage{amssymb}
\usepackage{graphicx}
\usepackage{multirow}
\usepackage{setspace}
%\usepackage{cite}
\usepackage{balance}
\usepackage{textcomp}
\usepackage{color}
\usepackage[ampersand]{easylist}
\ListProperties(Hide=100, Hang=true, Progressive=3ex, Style*=$\bullet$ ,
Style2*=\tiny$\blacksquare$ ,Style3*=$\circ$ ,Style4*=-- )
\usepackage{changepage}
\usepackage{algorithm}
\usepackage{algpseudocode}
\usepackage{graphics}
\usepackage{epsfig}
\floatname{algorithm}{Algorithm}
\renewcommand{\algorithmicrequire}{\textbf{Input:}}
\renewcommand{\algorithmicensure}{\textbf{Output:}}
\renewcommand{\algorithmiccomment}{// }
\usepackage{verbatim}
\newtheorem{theorem}{Theorem}
\newtheorem{lemma}{Lemma}
\newtheorem{prop}{Proposition}
\newtheorem{defn}{Definition}

\usepackage{eqparbox}
\usepackage{textcomp}

\renewcommand\algorithmiccomment[1]{%
  \hfill\#\ \eqparbox{COMMENT}{#1}%
}

\usepackage{verbatim}

\usepackage{indentfirst}

\usepackage{chngcntr}
\counterwithout{figure}{chapter}
\counterwithout{table}{chapter}

\usepackage{color}

% source code hightlighting
\usepackage{listings}
\lstset{
	numbers=left,
	stepnumber=1,
	firstnumber=1,
	captionpos=b,
	tabsize=2,
	basicstyle=\small,
	numberfirstline=true
}

% setting the page number to footer
\usepackage{fancyhdr}
\fancyhf{}
\cfoot{\thepage}
\pagestyle{fancy}
% no header and footer bar
\renewcommand{\headrulewidth}{0pt}
\renewcommand{\footrulewidth}{0pt}

% line height setting
\linespread{1.5}
\usepackage{setspace}

\usepackage{pdfpages}

% 浮水印套件 + 語法
\usepackage{eso-pic}
\newcommand\MyAtPageCenter[1]{\AtPageUpperLeft{%
\put(\LenToUnit{.42\paperwidth},\LenToUnit{-.5\paperheight}){#1}}%
}
% 浮水印套件 + 語法 end

% 調整標題格式
\usepackage{titlesec}
\usepackage{titletoc}

% 調整章節標題與頁面頂部的距離
\titlespacing*{\chapter}{0pt}{-30pt}{40pt}
% 參數說明:{左縮排}{頂部間距}{標題後間距}
% -30pt 表示向上移動 30pt(數字越小越靠近頂部)

% 自訂語法
\usepackage{etoolbox}
\newtoggle{toc-use-cn}
\newtoggle{iamphd}
\newtoggle{EnableDynTitle}

\usepackage{adjustbox}
\usepackage{xstring}
\usepackage{anyfontsize}

% book table語法
\usepackage{booktabs}

% 設定目錄的最大深度到 subsubsection 層級
\setcounter{tocdepth}{3}

% 控制編號的深度到 subsubsection 層級
\setcounter{secnumdepth}{3}
